


\section{Sustainability of experimental software tools}

Good software is defined not only by the quality of the things that you can make/do with it, but also how easily understandable the code base is, and how easy it is to extend or maintain. When experimenting and exploring, the rigidity of following "standard practice" may hamper creativity and slow down exploration in a way that is not beneficial to science. In this way, software is comparable to more creative endeavors like furniture design and visual arts. However, like other sciences, an experiment must be repeatable, otherwise the results do not mean anything. 


If a software is the experimental result of a paper, I would argue that it is an unsustainable and unscientific practice to write low quality code. If a software tool not only does not run but requires significant time and effort to refactor into a working state, then software as an experimental result of a study is not reproducible, and to a certain degree invalidates the study.

Other fields of science are dependent on the results presented to be followed by a structured and correct method that allows readers of the study to recreate the results to verify the results. Science is not built on trust, but verifiability and reproducibility.